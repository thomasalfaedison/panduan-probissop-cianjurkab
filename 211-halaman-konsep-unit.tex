\section*{5. Menu Konsep}
\addcontentsline{toc}{section}{5. Menu Konsep}

\noindent Untuk mengakses Halaman Menu Konsep, pada daftar menu sebelah kiri klik Menu pada menu SOP seperti gambar berikut:
\begin{figure}[H]
      \centering
      \includegraphics[width=12.45cm, height=5.35cm]{media/konsep/1-menu-konsep.png}
      \caption{Akses Menu Konsep}
      \label{fig:bagan-alir}
    \end{figure}
    \vspace{-2em}

Setelah diklik maka akan diarahkan ke halaman Menu Konsep seperti berikut:
\begin{figure}[H]
      \centering
      \includegraphics[width=12.45cm, height=5.35cm]{media/konsep/2-halaman-konsep.png}
      \caption{Halaman Menu Konsep}
      \label{fig:bagan-alir}
    \end{figure}
    \vspace{-2em}

Pada Menu Konsep, ada beberapa fitur yang user dapat gunakan seperti Export Data Konsep, Filter Data Konsep, kirim data untuk diperiksa, lihat detail data Konsep, ubah data Konsep, dan hapus data Konsep. Berikut tahapan setiap fitur pada Menu Konsep:

\subsection*{5.1 Tambah Konsep}
\addcontentsline{toc}{subsection}{5.1 Tambah Konsep}

Untuk melakukan tambah Konsep dapat dilakukan dengan klik tombol “Tambah SOP".
\begin{figure}[H]
    \centering
    \includegraphics[width=12.45cm, height=5.35cm]{media/konsep/1-tombol-tambah-sop.png}
    \caption{Tampilan Tombol Tambah Konsep}
    \label{fig:bagan-alir}
\end{figure}
\vspace{-2em}

Setelah klik tombol “Tambah SOP", akan dialihkan ke halaman form tambah Konsep. User akan diminta untuk mengisi Judul SOP, Nomor SOP, Tanggal Pembuatan, Tanggal Revisi, Tanggal Efektif, Kata Kunci, memilih SOP Level (Pilih Umum), memilih Status Arsip, dan pilih Lintas Fungsi.

\newpage

\subsection*{5.2 Export Konsep}
\addcontentsline{toc}{subsection}{5.2 Export Konsep}

Untuk Export Data Konsep, klik tombol Export Konsep, maka otomatis download Data Konsep.
\begin{figure}[H]
    \centering
    \includegraphics[width=12.45cm, height=5.75cm]{media/konsep/1-export-konsep.png}
    \caption{Export Konsep}
    \label{fig:bagan-alir}
\end{figure}
\vspace{-2em}

Berikut hasil Export Konsep.
\begin{figure}[H]
    \centering
    \includegraphics[width=12.45cm, height=5.75cm]{media/konsep/2-hasil-export-konsep.png}
    \caption{Hasil Export Konsep}
    \label{fig:bagan-alir}
\end{figure}

\newpage

\subsection*{5.3 Filter Konsep}
\addcontentsline{toc}{subsection}{5.3 Filter Konsep}

Untuk Filter Data Konsep, isi atau pilih dengan data yang diinginkan pada header table.
\begin{figure}[H]
    \centering
    \includegraphics[width=12.45cm, height=5.75cm]{media/konsep/3-filter-konsep.png}
    \caption{Filter Konsep}
    \label{fig:bagan-alir}
\end{figure}

\subsection*{5.4 Kirim Data Konsep}
\addcontentsline{toc}{subsection}{5.4 Kirim Data Konsep}

Untuk mengirimkan data agar diperiksa, klik Icon Pesawat Terbang. 
\begin{figure}[H]
    \centering
    \includegraphics[width=12.45cm, height=5.75cm]{media/konsep/4-icon-pesawat.png}
    \caption{Kirim Data SOP untuk Diperiksa}
    \label{fig:bagan-alir}
\end{figure}
\vspace{-2em}

Setelah diklik maka akan masuk ke daftar Data yang perlu diperiksa sebelum dilakukan Persetujuan atau Penolakan.

\newpage

\subsection*{5.5 Detail Data Konsep}
\addcontentsline{toc}{subsection}{5.5 Detail Data Konsep}

Untuk melihat detail Konsep, klik pada nama Konsep atau klik ikon mata pada kolom tabel dan akan diarahkan ke halaman detail. 
\begin{figure}[H]
    \centering
    \includegraphics[width=12.45cm, height=5.75cm]{media/konsep/5-tombol-detail-konsep.png}
    \caption{Akses Halaman Detail Konsep}
    \label{fig:bagan-alir}
\end{figure}
\vspace{-2em}

Setelah diklik maka akan diarahkan ke halaman detail Konsep. 
\begin{figure}[H]
    \centering
    \includegraphics[width=12.45cm, height=5.75cm]{media/konsep/6-halaman-detail-konsep.png}
    \label{fig:bagan-alir}
\end{figure}
\begin{figure}[H]
    \centering
    \includegraphics[width=12.45cm, height=5.75cm]{media/konsep/6-halaman-detail-konsep-2.png}
    \label{fig:bagan-alir}
\end{figure}
\begin{figure}[H]
    \centering
    \includegraphics[width=12.45cm, height=5.75cm]{media/konsep/6-halaman-detail-konsep-3.png}
    \caption{Halaman Detail Konsep}
    \label{fig:bagan-alir}
\end{figure}
\vspace{-2em}

Pada halaman detail Konsep, user ditampilkan informasi Konsep yang terkait dengan Konsep yang dipilih, Ubah Konsep, Export PDF, tambah Dasar Hukum, tambah Pelaksana, Tambah Aktivitas, dan melihat Diagram SOP.

Berikut penjelasan fitur pada halaman detail Konsep.

\begin{enumerate}
    \item \textbf{Ubah Konsep}\\
    Untuk mengubah Konsep, klik tombol "Ubah Konsep" dan setelah diklik maka akan diarahkan ke form ubah Konsep. Sesuaikan isian pada kolom isi di form Ubah Konsep.

    \item \textbf{Export PDF Konsep}\\
    Untuk Export Konsep, klik tombol Export Konsep, maka diarahkan ke halaman preview file PDF yang akan didownload.

    \item \textbf{Tambah Dasar Hukum}\\
    Untuk melakukan tambah Dasar Hukum dapat dilakukan dengan klik tombol “Tambah Dasar Hukum". Setelah klik tombol “Tambah Dasar Hukum”, maka akan dialihkan ke halaman form tambah Dasar Hukum. User akan diminta untuk mengisi Dasar Hukum. 

    \item \textbf{Tambah Kualifikasi Pelaksana}\\
    Untuk melakukan tambah Kualifikasi Pelaksana dapat dilakukan dengan klik tombol “Tambah Kualifikasi Pelaksana". Setelah klik tombol “Tambah Kualifikasi Pelaksana”, maka akan dialihkan ke halaman form tambah Kualifikasi Pelaksana. User akan diminta untuk mengisi Kualifikasi Pelaksana. 

    \item \textbf{Tambah Keterkaitan SOP}\\
    Untuk melakukan tambah Keterkaitan SOP dapat dilakukan dengan klik tombol “Tambah Keterkaitan SOP". Setelah klik tombol “Tambah Keterkaitan SOP”, maka akan dialihkan ke halaman form tambah Keterkaitan SOP. User akan diminta untuk mengisi Keterkaitan SOP. 

    \item \textbf{Tambah Peralatan/Perlengkapan}\\
    Untuk melakukan tambah Peralatan/Perlengkapan dapat dilakukan dengan klik tombol “Tambah Peralatan/Perlengkapan". Setelah klik tombol “Tambah Peralatan/Perlengkapan”, maka akan dialihkan ke halaman form tambah Peralatan/Perlengkapan. User akan diminta untuk mengisi Peralatan/Perlengkapan. 

    \item \textbf{Tambah Peringatan}\\
    Untuk melakukan tambah Peringatan dapat dilakukan dengan klik tombol “Tambah Peringatan". Setelah klik tombol “Tambah Peringatan”, maka akan dialihkan ke halaman form tambah Peringatan. User akan diminta untuk mengisi Peringatan. 

    \item \textbf{Tambah Pencatatan \& Pendataan}\\
    Untuk melakukan tambah Pencatatan \& Pendataan dapat dilakukan dengan klik tombol “Tambah Pencatatan \& Pendataan". Setelah klik tombol “Tambah Pencatatan \& Pendataan”, maka akan dialihkan ke halaman form tambah Pencatatan \& Pendataan. User akan diminta untuk mengisi Pencatatan \& Pendataan. 

    \item \textbf{Tambah Pengesahan}\\
    Untuk melakukan tambah Pengesahan dapat dilakukan dengan klik tombol “Tambah Pengesahan". Setelah klik tombol “Tambah Pengesahan”, maka akan dialihkan ke halaman form tambah Pengesahan. User akan diminta untuk mengisi Pengesahan. 

    \item \textbf{Tambah Pelaksana}\\
    Untuk melakukan tambah Pelaksana dapat dilakukan dengan klik tombol “Tambah Pelaksana”. Setelah klik tombol “Tambah Pelaksana”, maka akan dialihkan ke halaman form tambah Pelaksana. User  mengisi Urutan dan nama Pelaksana.

    \item \textbf{Tambah Aktivitas}\\
    Untuk melakukan tambah Aktivitas dapat dilakukan dengan klik tombol “Tambah Aktivitas”. Setelah klik tombol “Tambah Aktivitas”, maka akan dialihkan ke halaman form tambah Aktivitas. User  mengisi Urutan, pilih Diagram Alir, pilih Pelaksana, mengisi Aktivitas, mengisi Kelengkapan, mengisi Waktu, memilih Umum Kerja, mengisi Output, mengisi Keterangan, pilih Jika Tidak (Khusus Pengambilan Keputusan),  dan pilih Status Konektor.

\end{enumerate}

\subsection*{5.6 Ubah Data Konsep}
\addcontentsline{toc}{subsection}{5.6 Ubah Data Konsep}

Untuk mengubah data Konsep, bisa melalui halaman detail dengan klik tombol "Ubah SOP" atau klik Icon Pencil disebelah kanan Table pada halaman Konsep. 
\begin{figure}[H]
    \centering
    \includegraphics[width=12.45cm, height=5.75cm]{media/konsep/7-icon-pencil-konsep.png}
    \caption{Tombol Ubah Konsep}
    \label{fig:bagan-alir}
\end{figure}
\vspace{-2em}

Setelah diklik maka akan diarahkan ke form ubah Konsep. Sesuaikan isian pada kolom isi di form Ubah Konsep.

\newpage

\subsection*{5.6 Hapus Data Konsep}
\addcontentsline{toc}{subsection}{5.6 Hapus Data Konsep}

Untuk menghapus Konsep, klik Icon Sampah disebelah kanan Table dan akan muncul pesan konfirmasi untuk melakukan hapus data. 
\begin{figure}[H]
    \centering
    \includegraphics[width=12.45cm, height=4.75cm]{media/konsep/8-hapus-konsep.png}
    \caption{Tampilan Konfirmasi Hapus Konsep}
    \label{fig:bagan-alir}
\end{figure}