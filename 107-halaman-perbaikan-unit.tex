\section*{7. Menu Perbaikan}
\addcontentsline{toc}{section}{7. Menu Perbaikan}

\noindent Untuk mengakses Halaman Menu Perbaikan, pada daftar menu sebelah kiri klik Menu pada menu SOP seperti gambar berikut:
\begin{figure}[H]
      \centering
      \includegraphics[width=12.45cm]{media/Perbaikan-unit/1-menu-Perbaikan.png}
      \caption{Akses Menu Perbaikan}
      \label{fig:bagan-alir}
    \end{figure}
    \vspace{-2em}

Setelah diklik maka akan diarahkan ke halaman Menu Perbaikan seperti berikut:
\begin{figure}[H]
      \centering
      \includegraphics[width=12.45cm]{media/Perbaikan-unit/2-halaman-Perbaikan.png}
      \caption{Halaman Menu Perbaikan}
      \label{fig:bagan-alir}
    \end{figure}
    \vspace{-2em}

Pada Menu Perbaikan, ada beberapa fitur yang user dapat gunakan seperti Export Data Perbaikan, Filter Data Perbaikan. Berikut tahapan setiap fitur pada Menu Perbaikan:

\subsection*{7.1 Export Perbaikan}
\addcontentsline{toc}{subsection}{7.1 Export Perbaikan}

Untuk Export Data Perbaikan, klik tombol Export Perbaikan, maka otomatis download Data Perbaikan.
\begin{figure}[H]
    \centering
    \includegraphics[width=12.45cm]{media/Perbaikan-unit/1-export-Perbaikan.png}
    \caption{Export Perbaikan}
    \label{fig:bagan-alir}
\end{figure}
\vspace{-2em}

% Berikut hasil Export Perbaikan.
% \begin{figure}[H]
%     \centering
%     \includegraphics[width=12.45cm]{media/Perbaikan-unit/2-hasil-export-Perbaikan.png}
%     \caption{Hasil Export Perbaikan}
%     \label{fig:bagan-alir}
% \end{figure}

\newpage

\subsection*{7.2 Filter Perbaikan}
\addcontentsline{toc}{subsection}{7.2 Filter Perbaikan}

Untuk Filter Data Perbaikan, isi atau pilih dengan data yang diinginkan pada header table.
\begin{figure}[H]
    \centering
    \includegraphics[width=12.45cm]{media/Perbaikan-unit/3-filter-Perbaikan.png}
    \caption{Filter Perbaikan}
    \label{fig:bagan-alir}
\end{figure}

% \subsection*{7.3 Kirim Perbaikan Data}
% \addcontentsline{toc}{subsection}{7.3 Kirim Perbaikan Data}

% Untuk melakukan Kirim Perbaikan Data, klik Icon Ceklis. 
% \begin{figure}[H]
%     \centering
%     \includegraphics[width=12.45cm]{media/Perbaikan-unit/4-icon-ceklis.png}
%     \caption{Kirim Perbaikan Data SOP}
%     \label{fig:bagan-alir}
% \end{figure}
% \vspace{-2em}

% Setelah diklik maka akan masuk ke daftar Data yang akan diPerbaikan kembali oleh Admin.

% \newpage

% \subsection*{7.4 Detail Data Perbaikan}
% \addcontentsline{toc}{subsection}{7.4 Detail Data Perbaikan}

% Untuk melihat detail Perbaikan, klik pada nama Perbaikan atau klik ikon mata pada kolom tabel dan akan diarahkan ke halaman detail. 
% \begin{figure}[H]
%     \centering
%     \includegraphics[width=12.45cm]{media/Perbaikan-unit/5-tombol-detail-Perbaikan.png}
%     \caption{Akses Halaman Detail Perbaikan}
%     \label{fig:bagan-alir}
% \end{figure}
% \vspace{-2em}

% Setelah diklik maka akan diarahkan ke halaman detail Perbaikan. 
% \begin{figure}[H]
%     \centering
%     \includegraphics[width=12.45cm]{media/Perbaikan/6-halaman-detail-Perbaikan.png}
%     \label{fig:bagan-alir}
% \end{figure}
% \begin{figure}[H]
%     \centering
%     \includegraphics[width=12.45cm]{media/Perbaikan/6-halaman-detail-Perbaikan-2.png}
%     \label{fig:bagan-alir}
% \end{figure}
% \begin{figure}[H]
%     \centering
%     \includegraphics[width=12.45cm]{media/Perbaikan/6-halaman-detail-Perbaikan-3.png}
%     \caption{Halaman Detail Perbaikan}
%     \label{fig:bagan-alir}
% \end{figure}
% \vspace{-2em}

% Pada halaman detail Perbaikan, user ditampilkan informasi Perbaikan yang terkait dengan Perbaikan yang dipilih, Ubah Perbaikan, Export PDF, tambah Dasar Hukum, tambah Pelaksana, Tambah Aktivitas, dan melihat Diagram SOP.

% Berikut penjelasan fitur pada halaman detail Perbaikan.

% \begin{enumerate}
%     \item \textbf{Ubah Perbaikan}\\
%     Untuk mengubah Perbaikan, klik tombol "Ubah Perbaikan" dan setelah diklik maka akan diarahkan ke form ubah Perbaikan. Sesuaikan isian pada kolom isi di form Ubah Perbaikan.

%     \item \textbf{Export PDF Perbaikan}\\
%     Untuk Export Perbaikan, klik tombol Export Perbaikan, maka diarahkan ke halaman preview file PDF yang akan didownload.

%     \item \textbf{Tambah Dasar Hukum}\\
%     Untuk melakukan tambah Dasar Hukum dapat dilakukan dengan klik tombol “Tambah Dasar Hukum". Setelah klik tombol “Tambah Dasar Hukum”, maka akan dialihkan ke halaman form tambah Dasar Hukum. User akan diminta untuk mengisi Dasar Hukum. 

%     \item \textbf{Tambah Kualifikasi Pelaksana}\\
%     Untuk melakukan tambah Kualifikasi Pelaksana dapat dilakukan dengan klik tombol “Tambah Kualifikasi Pelaksana". Setelah klik tombol “Tambah Kualifikasi Pelaksana”, maka akan dialihkan ke halaman form tambah Kualifikasi Pelaksana. User akan diminta untuk mengisi Kualifikasi Pelaksana. 

%     \item \textbf{Tambah Keterkaitan SOP}\\
%     Untuk melakukan tambah Keterkaitan SOP dapat dilakukan dengan klik tombol “Tambah Keterkaitan SOP". Setelah klik tombol “Tambah Keterkaitan SOP”, maka akan dialihkan ke halaman form tambah Keterkaitan SOP. User akan diminta untuk mengisi Keterkaitan SOP. 

%     \item \textbf{Tambah Peralatan/Perlengkapan}\\
%     Untuk melakukan tambah Peralatan/Perlengkapan dapat dilakukan dengan klik tombol “Tambah Peralatan/Perlengkapan". Setelah klik tombol “Tambah Peralatan/Perlengkapan”, maka akan dialihkan ke halaman form tambah Peralatan/Perlengkapan. User akan diminta untuk mengisi Peralatan/Perlengkapan. 

%     \item \textbf{Tambah Peringatan}\\
%     Untuk melakukan tambah Peringatan dapat dilakukan dengan klik tombol “Tambah Peringatan". Setelah klik tombol “Tambah Peringatan”, maka akan dialihkan ke halaman form tambah Peringatan. User akan diminta untuk mengisi Peringatan. 

%     \item \textbf{Tambah Pencatatan \& Pendataan}\\
%     Untuk melakukan tambah Pencatatan \& Pendataan dapat dilakukan dengan klik tombol “Tambah Pencatatan \& Pendataan". Setelah klik tombol “Tambah Pencatatan \& Pendataan”, maka akan dialihkan ke halaman form tambah Pencatatan \& Pendataan. User akan diminta untuk mengisi Pencatatan \& Pendataan. 

%     \item \textbf{Tambah Pengesahan}\\
%     Untuk melakukan tambah Pengesahan dapat dilakukan dengan klik tombol “Tambah Pengesahan". Setelah klik tombol “Tambah Pengesahan”, maka akan dialihkan ke halaman form tambah Pengesahan. User akan diminta untuk mengisi Pengesahan. 

%     \item \textbf{Tambah Pelaksana}\\
%     Untuk melakukan tambah Pelaksana dapat dilakukan dengan klik tombol “Tambah Pelaksana”. Setelah klik tombol “Tambah Pelaksana”, maka akan dialihkan ke halaman form tambah Pelaksana. User  mengisi Urutan dan nama Pelaksana.

%     \item \textbf{Tambah Aktivitas}\\
%     Untuk melakukan tambah Aktivitas dapat dilakukan dengan klik tombol “Tambah Aktivitas”. Setelah klik tombol “Tambah Aktivitas”, maka akan dialihkan ke halaman form tambah Aktivitas. User  mengisi Urutan, pilih Diagram Alir, pilih Pelaksana, mengisi Aktivitas, mengisi Kelengkapan, mengisi Waktu, memilih Umum Kerja, mengisi Output, mengisi Keterangan, pilih Jika Tidak (Khusus Pengambilan Keputusan),  dan pilih Status Konektor.

% \end{enumerate}

% \subsection*{7.5 Perbaikan Data SOP}
% \addcontentsline{toc}{subsection}{7.5 Perbaikan Data SOP}

% Untuk perbaikan data SOP, bisa melalui halaman detail dengan klik tombol "Sunting Perbaikan" atau klik Icon Pencil disebelah kanan Table pada halaman Perbaikan. 
% \begin{figure}[H]
%     \centering
%     \includegraphics[width=12.45cm]{media/Perbaikan-unit/6-icon-pencil.png}
%     \caption{Tombol Ubah Perbaikan}
%     \label{fig:bagan-alir}
% \end{figure}

% Setelah diklik maka akan diarahkan ke form ubah Perbaikan. Sesuaikan isian pada kolom isi di form Ubah Perbaikan.

