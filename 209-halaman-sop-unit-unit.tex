\section*{3. Menu SOP Unit}
\addcontentsline{toc}{section}{3. Menu SOP Unit}

\noindent Untuk mengakses Halaman Menu SOP Unit, pada daftar menu sebelah kiri klik Menu pada menu SOP seperti gambar berikut:
\begin{figure}[H]
      \centering
      \includegraphics[width=12.45cm, height=5.35cm]{media/sop-unit-unit/1-menu-sop-unit.png}
      \caption{Akses Menu SOP Unit}
      \label{fig:bagan-alir}
    \end{figure}
    \vspace{-2em}

Setelah diklik maka akan diarahkan ke halaman Menu SOP Unit seperti berikut:
\begin{figure}[H]
      \centering
      \includegraphics[width=12.45cm, height=5.35cm]{media/sop-unit-unit/2-halaman-sop-unit.png}
      \caption{Halaman Menu SOP Unit}
      \label{fig:bagan-alir}
    \end{figure}
    \vspace{-2em}

Pada Menu SOP Unit, ada beberapa fitur yang user dapat gunakan seperti tambah SOP Unit, Export Data SOP Unit, Filter Data SOP Unit, dan lihat detail data SOP Unit. Berikut tahapan setiap fitur pada Menu SOP Unit:

\newpage

\subsection*{3.1 Tambah SOP Unit}
\addcontentsline{toc}{subsection}{3.1 Tambah SOP Unit}

Untuk melakukan tambah SOP Unit dapat dilakukan dengan klik tombol “Tambah SOP".
\begin{figure}[H]
    \centering
    \includegraphics[width=12.45cm, height=5.35cm]{media/sop-unit-unit/1-tombol-tambah.png}
    \caption{Tampilan Tombol Tambah SOP Unit}
    \label{fig:bagan-alir}
\end{figure}
\vspace{-2em}

Setelah klik tombol “Tambah SOP", akan dialihkan ke halaman form tambah SOP Unit. User akan diminta untuk mengisi Judul SOP, Nomor SOP, Tanggal Pembuatan, Tanggal Revisi, Tanggal Efektif, Kata Kunci, memilih SOP Level (Pilih Unit atau Umum), memilih Status Arsip (Ya atau Tidak), dan pilih Lintas Fungsi. Jika sudah terisi, lalu klik tombol simpan


\subsection*{3.2 Export SOP Unit}
\addcontentsline{toc}{subsection}{3.2 Export SOP Unit}

Untuk Export Data SOP Unit, klik tombol Export SOP Unit, maka otomatis download Data SOP Unit.
\begin{figure}[H]
    \centering
    \includegraphics[width=12.45cm, height=5.35cm]{media/sop-unit-unit/2-export-sop-unit.png}
    \caption{Export SOP Unit}
    \label{fig:bagan-alir}
\end{figure}
\vspace{-2em}

\newpage

Berikut hasil Export SOP Unit.
\begin{figure}[H]
    \centering
    \includegraphics[width=12.45cm, height=5.35cm]{media/sop-unit-unit/3-hasil-export.png}
    \caption{Hasil Export SOP Unit}
    \label{fig:bagan-alir}
\end{figure}

\subsection*{3.3 Filter SOP Unit}
\addcontentsline{toc}{subsection}{3.3 Filter SOP Unit}

Untuk Filter Data SOP Unit, isi atau pilih dengan data yang diinginkan pada header table.
\begin{figure}[H]
    \centering
    \includegraphics[width=12.45cm, height=5.35cm]{media/sop-unit-unit/4-filter-sop-unit.png}
    \caption{Filter SOP Unit}
    \label{fig:bagan-alir}
\end{figure}

\newpage

\subsection*{3.4 Kirim Data SOP Unit}
\addcontentsline{toc}{subsection}{3.4 Kirim Data SOP Unit}

Untuk melakukan Kirim SOP Unit, jika ingin mengirimkan data untuk diperiksa, klik Icon Pesawat Terbang. 
\begin{figure}[H]
    \centering
    \includegraphics[width=12.45cm, height=5.75cm]{media/sop-unit-unit/7-kirim-periksa.png}
    \caption{Kirim Data SOP untuk Diperiksa}
    \label{fig:bagan-alir}
\end{figure}
\vspace{-2em}

Setelah diklik maka akan masuk ke daftar Data yang perlu diperiksa sebelum dilakukan Persetujuan atau Penolakan oleh Admin.

\subsection*{3.5 Detail Data SOP Unit}
\addcontentsline{toc}{subsection}{3.5 Detail Data SOP Unit}

Untuk melihat detail SOP Unit, klik pada nama SOP Unit atau klik ikon mata pada kolom tabel dan akan diarahkan ke halaman detail. 
\begin{figure}[H]
    \centering
    \includegraphics[width=12.45cm, height=5.35cm]{media/sop-unit-unit/5-lihat-detail-sop-unit.png}
    \caption{Akses Halaman Detail SOP Unit}
    \label{fig:bagan-alir}
\end{figure}
\vspace{-2em}

Setelah diklik maka akan diarahkan ke halaman detail SOP Unit. 
\begin{figure}[H]
    \centering
    \includegraphics[width=12.45cm, height=5.35cm]{media/sop-unit-unit/6-halaman-detail-sop-unit.png}
    \label{fig:bagan-alir}
\end{figure}
\begin{figure}[H]
    \centering
    \includegraphics[width=12.45cm, height=5.35cm]{media/sop-unit-unit/6-halaman-detail-sop-unit-2.png}
    \label{fig:bagan-alir}
\end{figure}
\begin{figure}[H]
    \centering
    \includegraphics[width=12.45cm, height=4.85cm]{media/sop-unit-unit/6-halaman-detail-sop-unit-3.png}
    \caption{Halaman Detail SOP Unit}
    \label{fig:bagan-alir}
\end{figure}
\vspace{-2em}

Pada halaman detail SOP Umum, user ditampilkan informasi SOP Umum yang terkait dengan SOP Umum yang dipilih, Export PDF, dan melihat Diagram SOP. 

Jika data tersebut belum dikirim untuk diperiksa maka user dapat Ubah SOP Umum, tambah Dasar Hukum, Tambah Kualifikasi Pelaksana, Tambah Keterkaitan SOP, Tambah Peralatan/Perlengkapan, Tambah Peringatan, Tambah Pencatatan \& Pendataan, Tambah Pengesahan, tambah Pelaksana, dan Tambah Aktivitas. 

Berikut penjelasan fitur pada halaman detail SOP Umum.

\begin{enumerate}
    \item \textbf{Ubah SOP Umum}\\
    Untuk mengubah SOP Umum, klik tombol "Ubah SOP Umum" dan setelah diklik maka akan diarahkan ke form ubah SOP Umum. Sesuaikan isian pada kolom isi di form Ubah SOP Umum.

    \item \textbf{Export PDF SOP Umum}\\
    Untuk Export SOP Umum, klik tombol Export SOP Umum, maka diarahkan ke halaman preview file PDF yang akan didownload.

    \item \textbf{Tambah Dasar Hukum}\\
    Untuk melakukan tambah Dasar Hukum dapat dilakukan dengan klik tombol “Tambah Dasar Hukum". Setelah klik tombol “Tambah Dasar Hukum”, maka akan dialihkan ke halaman form tambah Dasar Hukum. User akan diminta untuk mengisi Dasar Hukum. 

    \item \textbf{Tambah Kualifikasi Pelaksana}\\
    Untuk melakukan tambah Kualifikasi Pelaksana dapat dilakukan dengan klik tombol “Tambah Kualifikasi Pelaksana". Setelah klik tombol “Tambah Kualifikasi Pelaksana”, maka akan dialihkan ke halaman form tambah Kualifikasi Pelaksana. User akan diminta untuk mengisi Kualifikasi Pelaksana. 

    \item \textbf{Tambah Keterkaitan SOP}\\
    Untuk melakukan tambah Keterkaitan SOP dapat dilakukan dengan klik tombol “Tambah Keterkaitan SOP". Setelah klik tombol “Tambah Keterkaitan SOP”, maka akan dialihkan ke halaman form tambah Keterkaitan SOP. User akan diminta untuk mengisi Keterkaitan SOP. 

    \item \textbf{Tambah Peralatan/Perlengkapan}\\
    Untuk melakukan tambah Peralatan/Perlengkapan dapat dilakukan dengan klik tombol “Tambah Peralatan/Perlengkapan". Setelah klik tombol “Tambah Peralatan/Perlengkapan”, maka akan dialihkan ke halaman form tambah Peralatan/Perlengkapan. User akan diminta untuk mengisi Peralatan/Perlengkapan. 

    \item \textbf{Tambah Peringatan}\\
    Untuk melakukan tambah Peringatan dapat dilakukan dengan klik tombol “Tambah Peringatan". Setelah klik tombol “Tambah Peringatan”, maka akan dialihkan ke halaman form tambah Peringatan. User akan diminta untuk mengisi Peringatan. 

    \item \textbf{Tambah Pencatatan \& Pendataan}\\
    Untuk melakukan tambah Pencatatan \& Pendataan dapat dilakukan dengan klik tombol “Tambah Pencatatan \& Pendataan". Setelah klik tombol “Tambah Pencatatan \& Pendataan”, maka akan dialihkan ke halaman form tambah Pencatatan \& Pendataan. User akan diminta untuk mengisi Pencatatan \& Pendataan. 

    \item \textbf{Tambah Pengesahan}\\
    Untuk melakukan tambah Pengesahan dapat dilakukan dengan klik tombol “Tambah Pengesahan". Setelah klik tombol “Tambah Pengesahan”, maka akan dialihkan ke halaman form tambah Pengesahan. User akan diminta untuk mengisi Pengesahan. 

    \item \textbf{Tambah Pelaksana}\\
    Untuk melakukan tambah Pelaksana dapat dilakukan dengan klik tombol “Tambah Pelaksana”. Setelah klik tombol “Tambah Pelaksana”, maka akan dialihkan ke halaman form tambah Pelaksana. User  mengisi Urutan dan nama Pelaksana.

    \item \textbf{Tambah Aktivitas}\\
    Untuk melakukan tambah Aktivitas dapat dilakukan dengan klik tombol “Tambah Aktivitas”. Setelah klik tombol “Tambah Aktivitas”, maka akan dialihkan ke halaman form tambah Aktivitas. User  mengisi Urutan, pilih Diagram Alir, pilih Pelaksana, mengisi Aktivitas, mengisi Kelengkapan, mengisi Waktu, memilih Umum Kerja, mengisi Output, mengisi Keterangan, pilih Jika Tidak (Khusus Pengambilan Keputusan),  dan pilih Status Konektor.

\end{enumerate}

\subsection*{3.6 Ubah Data SOP Unit}
\addcontentsline{toc}{subsection}{3.6 Ubah Data SOP Unit}

Jika data belum dikirim untuk diperiksa, user dapat melakukan ubah data SOP Unit melalui halaman detail dengan klik tombol "Ubah SOP" atau klik Icon Pencil disebelah kanan Table pada halaman SOP Unit. 
\begin{figure}[H]
    \centering
    \includegraphics[width=12.45cm, height=5.35cm]{media/sop-Unit-unit/8-icon-pencil.png}
    \caption{Tombol Ubah SOP Unit}
    \label{fig:bagan-alir}
\end{figure}

Setelah diklik maka akan diarahkan ke form ubah SOP Unit. Sesuaikan isian pada kolom isi di form Ubah SOP Unit.

\subsection*{3.7 Hapus Data SOP Unit}
\addcontentsline{toc}{subsection}{3.7 Hapus Data SOP Unit}

Jika data SOP Unit belum dikirim untuk diperiksa, user dapat melakukan hapus data SOP Unit dengan klik Icon Sampah disebelah kanan Table dan akan muncul pesan konfirmasi untuk melakukan hapus data. 
\begin{figure}[H]
    \centering
    \includegraphics[width=12.45cm, height=5.35cm]{media/sop-Unit-unit/9-hapus.png}
    \caption{Tampilan Konfirmasi Hapus SOP Unit}
    \label{fig:bagan-alir}
\end{figure}