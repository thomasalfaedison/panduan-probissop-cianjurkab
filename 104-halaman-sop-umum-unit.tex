\section*{4. Menu SOP Umum}
\addcontentsline{toc}{section}{4. Menu SOP Umum}

\noindent Untuk mengakses Halaman Menu SOP Umum, pada daftar menu sebelah kiri klik Menu pada menu SOP seperti gambar berikut:
\begin{figure}[H]
      \centering
      \includegraphics[width=12.45cm]{media/sop-umum-unit/1-menu-sop-umum.png}
      \caption{Akses Menu SOP Umum}
      \label{fig:bagan-alir}
    \end{figure}
    \vspace{-2em}

Setelah diklik maka akan diarahkan ke halaman Menu SOP Umum seperti berikut:
\begin{figure}[H]
      \centering
      \includegraphics[width=12.45cm]{media/sop-umum-unit/2-halaman-sop-umum.png}
      \caption{Halaman Menu SOP Umum}
      \label{fig:bagan-alir}
    \end{figure}
    \vspace{-2em}

Pada Menu SOP Umum, ada beberapa fitur yang user dapat gunakan seperti tambah SOP Umum, Export Data SOP Umum, Filter Data SOP Umum, Kirim untuk diperiksa, lihat detail data SOP Umum, ubah data SOP Umum, dan hapus data SOP Umum. Berikut tahapan setiap fitur pada Menu SOP Umum:

\newpage

\subsection*{4.1 Tambah SOP Umum}
\addcontentsline{toc}{subsection}{4.1 Tambah SOP Umum}

Untuk melakukan tambah SOP Umum dapat dilakukan dengan klik tombol “Tambah SOP".
\begin{figure}[H]
    \centering
    \includegraphics[width=12.45cm]{media/sop-umum-unit/1-tombol-tambah-umum.png}
    \caption{Tampilan Tombol Tambah SOP Umum}
    \label{fig:bagan-alir}
\end{figure}
\vspace{-2em}

Setelah klik tombol “Tambah SOP", akan dialihkan ke halaman form tambah SOP Umum. Untuk Form SOP Umum, secara tampilan dan kolom isian masih sama seperti Form Tambah SOP Unit seperti akan diminta untuk mengisi Judul SOP, Nomor SOP, Tanggal Pembuatan, Tanggal Revisi, Tanggal Efektif, Kata Kunci, memilih SOP Level (Untuk SOP Umum maka pilih Umum), memilih Status Arsip, dan pilih Lintas Fungsi.

\subsection*{4.2 Export SOP Umum}
\addcontentsline{toc}{subsection}{4.2 Export SOP Umum}

Untuk Export Data SOP Umum, klik tombol Export SOP Umum, maka otomatis download Data SOP Umum.
\begin{figure}[H]
    \centering
    \includegraphics[width=12.45cm]{media/sop-umum-unit/2-export-sop-umum.png}
    \caption{Export SOP Umum}
    \label{fig:bagan-alir}
\end{figure}
\vspace{-2em}

Berikut hasil Export SOP Umum.
\begin{figure}[H]
    \centering
    \includegraphics[width=12.45cm]{media/sop-umum-unit/3-hasil-export.png}
    \caption{Hasil Export SOP Umum}
    \label{fig:bagan-alir}
\end{figure}

\subsection*{4.3 Filter SOP Umum}
\addcontentsline{toc}{subsection}{4.2 Filter SOP Umum}

Untuk Filter Data SOP Umum, isi atau pilih dengan data yang diinginkan pada header table.
\begin{figure}[H]
    \centering
    \includegraphics[width=12.45cm]{media/sop-umum-unit/4-filter-sop-umum.png}
    \caption{Filter SOP Umum}
    \label{fig:bagan-alir}
\end{figure}

\subsection*{4.4 Kirim Data SOP Umum}
\addcontentsline{toc}{subsection}{4.4 Kirim Data SOP Umum}

Untuk melakukan Kirim SOP Umum, jika ingin mengirimkan data untuk diperiksa, klik Icon Pesawat Terbang. 
\begin{figure}[H]
    \centering
    \includegraphics[width=12.45cm]{media/sop-umum-unit/5-icon-pesawat.png}
    \caption{Kirim Data SOP untuk Diperiksa}
    \label{fig:bagan-alir}
\end{figure}
\vspace{-2em}

Setelah diklik maka akan masuk ke daftar Data yang perlu diperiksa sebelum dilakukan Persetujuan atau Penolakan oleh Admin.

\newpage

\subsection*{4.5 Detail Data SOP Umum}
\addcontentsline{toc}{subsection}{4.5 Detail Data SOP Umum}

Untuk melihat detail SOP Umum, klik pada nama SOP Umum atau klik ikon mata pada kolom tabel dan akan diarahkan ke halaman detail. 
\begin{figure}[H]
    \centering
    \includegraphics[width=12.45cm, height=4.75cm]{media/sop-umum-unit/6-detail-sop-umum.png}
    \caption{Akses Halaman Detail SOP Umum}
    \label{fig:bagan-alir}
\end{figure}
\vspace{-2em}

Setelah diklik maka akan diarahkan ke halaman detail SOP Umum. 
\begin{figure}[H]
    \centering
    \includegraphics[width=12.45cm]{media/sop-Umum-unit/8-halaman-detail-sop-Umum.png}
    \label{fig:bagan-alir}
\end{figure}
\begin{figure}[H]
    \centering
    \includegraphics[width=12.45cm]{media/sop-Umum-unit/8-halaman-detail-sop-Umum-2.png}
    \label{fig:bagan-alir}
\end{figure}
\begin{figure}[H]
    \centering
    \includegraphics[width=12.45cm]{media/sop-Umum-unit/8-halaman-detail-sop-Umum-3.png}
    \caption{Halaman Detail SOP Umum}
    \label{fig:bagan-alir}
\end{figure}
\vspace{-2em}

Pada halaman detail SOP Umum, user ditampilkan informasi SOP Umum yang terkait dengan SOP Umum yang dipilih, Export PDF, dan melihat Diagram SOP. 

Jika data tersebut belum dikirim untuk diperiksa maka user dapat Ubah SOP Umum, tambah Dasar Hukum, Tambah Kualifikasi Pelaksana, Tambah Keterkaitan SOP, Tambah Peralatan/Perlengkapan, Tambah Peringatan, Tambah Pencatatan \& Pendataan, Tambah Pengesahan, tambah Pelaksana, dan Tambah Aktivitas. 

Berikut penjelasan fitur pada halaman detail SOP Umum.

\begin{enumerate}
    \item \textbf{Ubah SOP Umum}\\
    Untuk mengubah SOP Umum, klik tombol "Ubah SOP Umum" dan setelah diklik maka akan diarahkan ke form ubah SOP Umum. Sesuaikan isian pada kolom isi di form Ubah SOP Umum.

    \item \textbf{Export PDF SOP Umum}\\
    Untuk Export SOP Umum, klik tombol Export SOP Umum, maka diarahkan ke halaman preview file PDF yang akan didownload. Untuk hasil Export sama seperti hasil Export untuk SOP Unit.

    \item \textbf{Tambah Dasar Hukum}\\
    Untuk melakukan tambah Dasar Hukum dapat dilakukan dengan klik tombol “Tambah Dasar Hukum". Setelah klik tombol “Tambah Dasar Hukum”, maka akan dialihkan ke halaman form tambah Dasar Hukum. Untuk tampilan dan kolom isian Form Tambah Dasar Hukum masih sama seperti Form Tambah Dasar Hukum pada Detail SOP Unit. User akan diminta untuk mengisi Dasar Hukum. 

    \item \textbf{Tambah Pelaksana}\\
    Untuk melakukan tambah Pelaksana dapat dilakukan dengan klik tombol “Tambah Pelaksana”. Setelah klik tombol “Tambah Pelaksana”, maka akan dialihkan ke halaman form tambah Pelaksana. Untuk tampilan dan kolom isian Form Tambah Pelaksana masih sama seperti Form Tambah Pelaksana pada detail SOP Unit. User  mengisi Urutan dan nama Pelaksana.

    \item \textbf{Tambah Aktivitas}\\
    Untuk melakukan tambah Aktivitas dapat dilakukan dengan klik tombol “Tambah Aktivitas”. Setelah klik tombol “Tambah Aktivitas”, maka akan dialihkan ke halaman form tambah Aktivitas. Untuk tampilan dan kolom isian Form Tambah Aktivitas masih sama seperti Form Tambah Aktivitas di Detail SOP Unit. User  mengisi Urutan, pilih Diagram Alir, pilih Pelaksana, mengisi Aktivitas, mengisi Kelengkapan, mengisi Waktu, memilih Umum Kerja, mengisi Output, mengisi Keterangan, pilih Jika Tidak (Khusus Pengambilan Keputusan), dan pilih Status Konektor.

\end{enumerate}

\subsection*{4.6 Ubah Data SOP Umum}
\addcontentsline{toc}{subsection}{4.6 Ubah Data SOP Umum}

Jika data SOP Umum belum dikirim untuk diperiksa, user dapat melakukan ubah data SOP Umum. Untuk mengubah data SOP Umum, bisa melalui halaman detail dengan klik tombol "Sunting SOP Umum" atau klik Icon Pencil disebelah kanan Table pada halaman SOP Umum. 
\begin{figure}[H]
    \centering
    \includegraphics[width=12.45cm]{media/sop-umum-unit/7-icon-pencil.png}
    \caption{Tombol Ubah SOP Umum}
    \label{fig:bagan-alir}
\end{figure}

Setelah diklik maka akan diarahkan ke form ubah SOP Umum. Untuk tampilan dan kolom isian Form Ubah SOP masih sama seperti Form Ubah SOP Unit. User menyesuaikan isian pada kolom isi di form Ubah SOP Umum.

\newpage

\subsection*{4.7 Hapus Data SOP Umum}
\addcontentsline{toc}{subsection}{4.7 Hapus Data SOP Umum}

Jika data SOP Umum belum dikirim untuk diperiksa, user dapat melakukan hapus data SOP Umum. Untuk menghapus SOP Umum, klik Icon Sampah disebelah kanan Table dan akan muncul pesan konfirmasi untuk melakukan hapus data. 
\begin{figure}[H]
    \centering
    \includegraphics[width=12.45cm]{media/sop-umum-unit/8-hapus-sop-umum.png}
    \caption{Tampilan Konfirmasi Hapus SOP Umum}
    \label{fig:bagan-alir}
\end{figure}