\section*{10. Menu Arsip}
\addcontentsline{toc}{section}{10. Menu Arsip}

\noindent Untuk mengakses Halaman Menu Arsip, pada daftar menu sebelah kiri klik Menu pada menu SOP seperti gambar berikut:
\begin{figure}[H]
      \centering
      \includegraphics[width=12.45cm]{media/Arsip-unit/1-menu-Arsip.png}
      \caption{Akses Menu Arsip}
      \label{fig:bagan-alir}
    \end{figure}
    \vspace{-2em}

Setelah diklik maka akan diarahkan ke halaman Menu Arsip seperti berikut:
\begin{figure}[H]
      \centering
      \includegraphics[width=12.45cm]{media/Arsip-unit/2-halaman-Arsip.png}
      \caption{Halaman Menu Arsip}
      \label{fig:bagan-alir}
    \end{figure}
    \vspace{-2em}

Pada Menu Arsip, ada beberapa fitur yang user dapat gunakan seperti tambah Arsip, Export Data Arsip, Filter Data Arsip, Kirim Data, lihat detail data Arsip, ubah data Arsip, dan hapus data Arsip. Berikut tahapan setiap fitur pada Menu Arsip:

\newpage

\subsection*{10.1 Tambah Arsip}
\addcontentsline{toc}{subsection}{10.1 Tambah Arsip}

Untuk melakukan tambah Arsip dapat dilakukan dengan klik tombol “Tambah SOP".
\begin{figure}[H]
    \centering
    \includegraphics[width=12.45cm]{media/arsip-unit/1-tombol-tambah-arsip.png}
    \caption{Tampilan Tombol Tambah Arsip}
    \label{fig:bagan-alir}
\end{figure}
\vspace{-2em}

Setelah klik tombol “Tambah SOP", akan dialihkan ke halaman form tambah SOP Arsip. Untuk Form SOP Arsip, secara tampilan dan kolom isian masih sama seperti Form Tambah SOP Unit seperti akan diminta untuk mengisi Judul SOP, Nomor SOP, Tanggal Pembuatan, Tanggal Revisi, Tanggal Efektif, Kata Kunci, memilih SOP Level, memilih Status Arsip (Jika SOP Arsip maka pilih Ya), dan pilih Lintas Fungsi.

\subsection*{10.2 Export Arsip}
\addcontentsline{toc}{subsection}{10.2 Export Arsip}

Untuk Export Data Arsip, klik tombol Export Arsip, maka otomatis download Data Arsip.
\begin{figure}[H]
    \centering
    \includegraphics[width=12.45cm]{media/arsip-unit/2-export-arsip.png}
    \caption{Export Arsip}
    \label{fig:bagan-alir}
\end{figure}
\vspace{-2em}

Berikut hasil Export Arsip.
\begin{figure}[H]
    \centering
    \includegraphics[width=12.45cm]{media/arsip-unit/3-hasil-export-arsip.png}
    \caption{Hasil Export Arsip}
    \label{fig:bagan-alir}
\end{figure}

\subsection*{10.3 Filter Arsip}
\addcontentsline{toc}{subsection}{10.2 Filter Arsip}

Untuk Filter Data Arsip, isi atau pilih dengan data yang diinginkan pada header table.
\begin{figure}[H]
    \centering
    \includegraphics[width=12.45cm]{media/arsip-unit/4-filter-arsip.png}
    \caption{Filter Arsip}
    \label{fig:bagan-alir}
\end{figure}

\subsection*{10.4 Kirim Data Arsip}
\addcontentsline{toc}{subsection}{10.4 Verifikasi Data Arsip}

Jika pada Arsip terdapat data dengan Status Proses dan perlu dikirim untuk diperiksa, klik Icon Pesawat Terbang. 
\begin{figure}[H]
    \centering
    \includegraphics[width=12.45cm]{media/arsip-unit/5-icon-pesawat.png}
    \caption{Kirim Data SOP untuk Diperiksa}
    \label{fig:bagan-alir}
\end{figure}
\vspace{-2em}

Setelah diklik maka akan masuk ke daftar Data yang perlu diperiksa sebelum dilakukan PerArsipan atau Penolakan.

\newpage

\subsection*{10.5 Detail Data Arsip}
\addcontentsline{toc}{subsection}{10.5 Detail Data Arsip}

Untuk melihat detail Arsip, klik pada nama Arsip atau klik ikon mata pada kolom tabel dan akan diarahkan ke halaman detail. 
\begin{figure}[H]
    \centering
    \includegraphics[width=12.45cm]{media/arsip-unit/6-tombol-detail-arsip.png}
    \caption{Akses Halaman Detail Arsip}
    \label{fig:bagan-alir}
\end{figure}
\vspace{-2em}

Setelah diklik maka akan diarahkan ke halaman detail Arsip. 
\begin{figure}[H]
    \centering
    \includegraphics[width=12.45cm]{media/arsip-unit/8-halaman-detail-arsip.png}
    \label{fig:bagan-alir}
\end{figure}
\vspace{-2em}
\begin{figure}[H]
    \centering
    \includegraphics[width=12.45cm]{media/arsip-unit/8-halaman-detail-arsip-2.png}
    \label{fig:bagan-alir}
\end{figure}
\vspace{-2em}
\begin{figure}[H]
    \centering
    \includegraphics[width=12.45cm]{media/arsip-unit/8-halaman-detail-arsip-3.png}
    \caption{Halaman Detail Arsip}
    \label{fig:bagan-alir}
\end{figure}
\vspace{-2em}

Pada halaman detail Arsip, user ditampilkan informasi Arsip yang terkait dengan Arsip yang dipilih, Ubah Arsip, Export PDF, tambah Dasar Hukum, tambah Pelaksana, Tambah Aktivitas, dan melihat Diagram SOP.

Berikut penjelasan fitur pada halaman detail Arsip.

\begin{enumerate}
    \item \textbf{Ubah SOP Umum}\\
    Untuk mengubah SOP Umum, klik tombol "Ubah SOP Umum" dan setelah diklik maka akan diarahkan ke form ubah SOP Umum. Sesuaikan isian pada kolom isi di form Ubah SOP Umum.

    \item \textbf{Export PDF SOP Umum}\\
    Untuk Export SOP Umum, klik tombol Export SOP Umum, maka diarahkan ke halaman preview file PDF yang akan didownload. Untuk hasil Export sama seperti hasil Export untuk SOP Unit.

    \item \textbf{Tambah Dasar Hukum}\\
    Untuk melakukan tambah Dasar Hukum dapat dilakukan dengan klik tombol “Tambah Dasar Hukum". Setelah klik tombol “Tambah Dasar Hukum”, maka akan dialihkan ke halaman form tambah Dasar Hukum. Untuk tampilan dan kolom isian Form Tambah Dasar Hukum masih sama seperti Form Tambah Dasar Hukum pada Detail SOP Unit. User akan diminta untuk mengisi Dasar Hukum. 

    \item \textbf{Tambah Pelaksana}\\
    Untuk melakukan tambah Pelaksana dapat dilakukan dengan klik tombol “Tambah Pelaksana”. Setelah klik tombol “Tambah Pelaksana”, maka akan dialihkan ke halaman form tambah Pelaksana. Untuk tampilan dan kolom isian Form Tambah Pelaksana masih sama seperti Form Tambah Pelaksana pada detail SOP Unit. User  mengisi Urutan dan nama Pelaksana.

    \item \textbf{Tambah Aktivitas}\\
    Untuk melakukan tambah Aktivitas dapat dilakukan dengan klik tombol “Tambah Aktivitas”. Setelah klik tombol “Tambah Aktivitas”, maka akan dialihkan ke halaman form tambah Aktivitas. Untuk tampilan dan kolom isian Form Tambah Aktivitas masih sama seperti Form Tambah Aktivitas di Detail SOP Unit. User  mengisi Urutan, pilih Diagram Alir, pilih Pelaksana, mengisi Aktivitas, mengisi Kelengkapan, mengisi Waktu, memilih Umum Kerja, mengisi Output, mengisi Keterangan, pilih Jika Tidak (Khusus Pengambilan Keputusan), dan pilih Status Konektor.

\end{enumerate}

Jika data dengan status Arsip, maka pada halaman Detail SOP tidak akan menampilkan fitur diatas.

\subsection*{10.6 Ubah Data Arsip}
\addcontentsline{toc}{subsection}{10.6 Ubah Data Arsip}

Untuk mengubah data Arsip, bisa melalui halaman detail dengan klik tombol "Sunting Arsip" atau klik Icon Pencil disebelah kanan Table pada halaman Arsip. 
\begin{figure}[H]
    \centering
    \includegraphics[width=12.45cm]{media/arsip-unit/7-icon-pencil.png}
    \caption{Tombol Ubah Arsip}
    \label{fig:bagan-alir}
\end{figure}

Setelah diklik maka akan diarahkan ke form ubah Arsip. Untuk tampilan dan kolom isian Form Ubah SOP masih sama seperti Form Ubah SOP Unit. Sesuaikan isian pada kolom isi di form Ubah Arsip.

\newpage

\subsection*{10.7 Hapus Data Arsip}
\addcontentsline{toc}{subsection}{10.7 Hapus Data Arsip}

Untuk menghapus Arsip, klik Icon Sampah disebelah kanan Table dan akan muncul pesan konfirmasi untuk melakukan hapus data. 
\begin{figure}[H]
    \centering
    \includegraphics[width=12.45cm]{media/arsip-unit/8-hapus-arsip.png}
    \caption{Tampilan Konfirmasi Hapus Arsip}
    \label{fig:bagan-alir}
\end{figure}